\documentclass{article}

%%%%%%%%%%%%%%
% Taken from https://tex.stackexchange.com/questions/83085/how-to-improve-listings-display-of-json-files/83100#83100
%%%%%%%%%%%%%%
\usepackage{bera}% optional: just to have a nice mono-spaced font
\usepackage{listings}
\usepackage{xcolor}

\colorlet{punct}{red!60!black}
\definecolor{background}{HTML}{EEEEEE}
\definecolor{delim}{RGB}{20,105,176}
\colorlet{numb}{magenta!60!black}

\lstdefinelanguage{json}{
    basicstyle=\normalfont\ttfamily,
    numbers=left,
    numberstyle=\scriptsize,
    stepnumber=1,
    numbersep=8pt,
    showstringspaces=false,
    breaklines=true,
    frame=lines,
    backgroundcolor=\color{background},
    literate=
     *{0}{{{\color{numb}0}}}{1}
      {1}{{{\color{numb}1}}}{1}
      {2}{{{\color{numb}2}}}{1}
      {3}{{{\color{numb}3}}}{1}
      {4}{{{\color{numb}4}}}{1}
      {5}{{{\color{numb}5}}}{1}
      {6}{{{\color{numb}6}}}{1}
      {7}{{{\color{numb}7}}}{1}
      {8}{{{\color{numb}8}}}{1}
      {9}{{{\color{numb}9}}}{1}
      {:}{{{\color{punct}{:}}}}{1}
      {,}{{{\color{punct}{,}}}}{1}
      {\{}{{{\color{delim}{\{}}}}{1}
      {\}}{{{\color{delim}{\}}}}}{1}
      {[}{{{\color{delim}{[}}}}{1}
      {]}{{{\color{delim}{]}}}}{1},
}
%%%%%%%%%%%%%%
\usepackage[colorlinks=true,linkcolor=blue,citecolor=yellow,filecolor=magenta,urlcolor=cyan]{hyperref}
\newcommand{\cymisversion}{0.1}

\begin{document}
\thispagestyle{empty}

\begin{center}
\huge\underline{CYMIS v\cymisversion\space Specification}
\end{center}
\tableofcontents
\clearpage
\pagenumbering{arabic} 

\section{Overview}
CYMIS (CYber sleuth Mod Installation Script) documents are json-based files that can generate customised mod versions from a set of input data. This is achieved by:
\begin{itemize}
\item Generating a set of boolean flags, intended to be done \textit{via} an installation wizard;
\item Generating the required mod files based on the values of the boolean flags.
\end{itemize}
The details of how a CYMIS document can be built to satisfy these requirements are given in section \ref{section:specification}. This section focusses on the concepts the CYMIS format is built upon, rather than the specification details.

\subsection{Components}
A CYMIS implementation is made up of two components --- a \textbf{wizard} and an \textbf{installer}. The job of the \textbf{wizard} is to generate a set of booleans. The job of the \textbf{installer} is to decide which files to include in the final mod build based on the values of those flags.

\subsection{Flags and the Flag Table}
A CYMIS implementation stores a list of defined flags in a \textbf{Flag Table}. These flags are stored under a name and are boolean-valued.\\
The two parts of a CYMIS implementation -- the \textbf{wizard} and the \textbf{installer} -- interact with the Flag Table in the following ways:
\begin{itemize}
\item The Wizard \textbf{creates} and \textbf{modifies the values} of flags;
\item The Installer \textbf{reads} flags and \textbf{decides} what to do based on their values.
\end{itemize}

\subsection{Important Files}
There are three critical components to a CYMIS-ready mod:
\begin{itemize}
\item a ``modfiles" folder,
\item a ``modoptions" folder,
\item an INSTALL.json file.
\end{itemize}
The ``modfiles" folder should be \textbf{empty}. It is the location in which the mod will be built.\\
The ``modoptions" folder contains the data that the installer can install into the ``modfiles" folder, based on the flags set in the CYMIS wizard.\\
The ``INSTALL.json" file contains the CYMIS document. The information that must go into this file is defined in section \ref{section:specification}.
\newpage

\section{Specification}\label{section:specification}
\subsection{Top Level}
The top level of the CYMIS script must contain the following three labels:
\begin{itemize}
\item Version
\item Wizard
\item Install
\end{itemize}
``Version" states which CYMIS version the script should be interpreted as. If the script is written to satisfy the stipulations of this document, that version should be \textbf{\cymisversion}.\\
``Wizard" is a list of pages that make up the installation wizard, specified in section \ref{section:wizardscript}. These pages contain boolean flags that are used to determine which mod files are used and/or generated.\\
``Install" is a list of possible routes the installation process could take, depending on which flags have been set in the wizard. This is specified in section \ref{section:installscript}.\\\\
The top level of a CYMIS document will therefore look like the following:
\begin{lstlisting}[language=json,firstnumber=1]
{
    "Version": 0.1
    "Wizard": [],
    "Install": []
}
\end{lstlisting}
\newpage
\subsection{Wizard Script}\label{section:wizardscript}
The Wizard is defined as a series of pages. Each page is defined by a dictionary containing the following items:
\begin{itemize}
\item Title
\item Contents
\item Flags
\end{itemize}
``Title" is the title of a page in the wizard.\\
``Contents" is the text that is displayed after the title, intended to describe what the current installer page does or what its purpose is.
``Flags" is a list of flags that can be set on the page and will be displayed as UI elements. The definitions of these flags are given in section \ref{section:flagtypes}.\\\\
The contents of the ``Wizard" item of the CYMIS document will therefore look like the following:
\begin{lstlisting}[language=json,firstnumber=1]
...
    "Wizard": 
    [
        {
            "Title": "The First Page",
            "Contents": "This is the first page.",
            "Flags": []        
        },
        {
            "Title": "The Second Page",
            "Contents": "This is the second page.",
            "Flags": []
        },
        ...
    ],
...
\end{lstlisting}
\newpage

\subsection{Flag Types}\label{section:flagtypes}
The CYMIS specification defines a number of flag types that can be used. Each flag type can itself potentially define more than a single flag, depending on the implementation. Each flag type must contain at a minimum a ``Type" keyword, which defines which flag type it is, and some way of providing names for flags to be entered into the Flag Table.
\subsubsection{Flag}
The ``Flag" flag type is intended to be implemented as a labelled checkbox, taking True or False values. The specification can contain the following items, with optional values in square brackets:
\begin{itemize}
\item Type
\item Name
\item Description
\item {[Default]}
\end{itemize}
``Type" should be set to \textbf{``Flag"}.\\
``Name" is the name this flag is referred to by.\\
``Description" is the label attached to the checkbox.\\
``Default" defines whether the Flag begins in a True or False state. By default, it is False.\\\\
Some example flag definitions are given below.
\begin{lstlisting}[language=json,firstnumber=1]
{
    "Type": "Flag",
    "Name": "False Flag 1",
    "Description": "This Flag is False by default."
}
\end{lstlisting}
\begin{lstlisting}[language=json,firstnumber=1]
{
    "Type": "Flag",
    "Name": "False Flag 2",
    "Description": "This Flag is also False by default.",
    "Default": false
}
\end{lstlisting}
\newpage
\begin{lstlisting}[language=json,firstnumber=1]
{
    "Type": "Flag",
    "Name": "True Flag 1",
    "Description": "This Flag is True by default.",
    "Default": true
}
\end{lstlisting}

\subsubsection{HiddenFlag}
The ``HiddenFlag" flag type is not intended to be displayed in the UI. It should behave like a flag, but without the necessity for a ``Description" item in the specification. The specification can contain the following items, with optional values in square brackets:
\begin{itemize}
\item Type
\item Name
\item {[Default]}
\end{itemize}
``Type" should be set to \textbf{``HiddenFlag"}.\\
``Name" is the name this flag is referred to by.\\
``Default" defines whether the Flag begins in a True or False state. By default, it is False.\\\\
Some example flag definitions are given below.
\begin{lstlisting}[language=json,firstnumber=1]
{
    "Type": "HiddenFlag",
    "Name": "False HiddenFlag 1",
}
\end{lstlisting}
\begin{lstlisting}[language=json,firstnumber=1]
{
    "Type": "HiddenFlag",
    "Name": "False HiddenFlag 2",
    "Default": false
}
\end{lstlisting}

\begin{lstlisting}[language=json,firstnumber=1]
{
    "Type": "Flag",
    "Name": "True HiddenFlag 1",
    "Default": true
}
\end{lstlisting}

\subsubsection{ChooseOne}
The ``ChooseOne" flag type is intended to be displayed as a group of mutually exclusive checkable UI elements, such as a list of radio buttons. This flag type requires neither a Name nor Description, since it is a group of flags. However, the flags defined by this flag type should themselves each have a Name and Description. The specification can contain the following items, with optional values in square brackets:
\begin{itemize}
\item Type
\item Flags
\item {[Default]}
\end{itemize}
``Type" should be set to \textbf{``ChooseOne"}.\\
``Flags" is a list of flags provided by the ChooseOne group.\\
``Default" is the name of the flag that is selected at the start. By default, it is the first flag.\\\\
Each flag in the ``Flags" item should contain the following items:
\begin{itemize}
\item Name
\item Description
\end{itemize}
``Name" is the name this flag is referred to by.\\
``Default" defines whether the Flag begins in a True or False state. By default, it is False.\\\\
An example ChooseOne definition is given below.
\newpage
\begin{lstlisting}[language=json,firstnumber=1]
{
    "Type": "ChooseOne",
    "Flags":
    [
        {
            "Name": "ExampleFlag1",
            "Description": "This is the first option."
        },
        {
            "Name": "ExampleFlag2",
            "Description": "This is the second option."
        },
        {
            "Name": "ExampleFlag3",
            "Description": "This is the third option."
        }
    ],
    "Default": "ExampleFlag2"
}
\end{lstlisting}
\newpage
\subsection{Install Script}\label{section:installscript}
The Install section of the CYMIS document describes how to build the data in the ``modfiles" folder, given a set of possible build options and flags stating which paths should be followed. This follows a simple if-then format, where a single flag -- or combinations thereof -- are entered into the ``if" section, and the build option is defined in the ``then" section. Note that the ``if" element is optional, and can be omitted if a certain build path should always be followed.\\\\
An ``if" statement can be built from nested single-entry dictionaries that resolve down to flags and condition operations. The most basic ``if" statement is a single flag:
\begin{lstlisting}[language=json,firstnumber=1]
"if": "Flag Name"
\end{lstlisting}
A more complicated ``if" statement will contain a rule and an argument to be passed to the rule:
\begin{lstlisting}[language=json,firstnumber=1]
"if": {<rule name>: <rule argument>}
\end{lstlisting}
For example, checking if multiple flags are true:
\begin{lstlisting}[language=json,firstnumber=1]
"if": {"and": ["Flag 1", "Flag 2"]}
\end{lstlisting}
or if one flag is true and another is false:
\begin{lstlisting}[language=json,firstnumber=1]
"if": {"and": ["Flag 1", {"not": "Flag 2"}]}
\end{lstlisting}
Complex ``if" statements can be built up in this manner. Any flag name can be replaced with a rule and argument in order to create arbitrarily deeply-nested conditions.\\\\
Build options, on the other hand, are simply a list of rules-and-arguments. The rules that can be used in a build option are detailed in section \ref{section:installationrules}.
\newpage
\noindent The contents of the ``Install" item of the CYMIS document will therefore look like the following:
\begin{lstlisting}[language=json,firstnumber=1]
...
    "Install": 
    [
        {
            "if": <condition>,
            "then": []   
        },
        {
            "if": <condition>,
            "then": []   
        },
        {
            "then": []   
        },
        ...
    ],
...
\end{lstlisting}
\newpage
\subsection{Condition Rules}\label{section:conditionrules}
In the following section, the word ``condition" is intended to mean either a \textbf{flag name} or a \textbf{condition rule plus its argument}, since both are valid values to pass to a condition rule.
\subsubsection{and}
The ``and" rule takes a \textbf{list} of conditions and returns True if \textbf{all} the contained conditions are true.
\begin{lstlisting}[language=json,firstnumber=1]
"if": {"and": [<condition 1>, <condition 2>, <condition 3>, ..., <condition N>]}
\end{lstlisting}
\subsubsection{or}
The ``or" rule takes a \textbf{list} of conditions and returns True if \textbf{any} the contained conditions are true.
\begin{lstlisting}[language=json,firstnumber=1]
"if": {"any": [<condition 1>, <condition 2>, <condition 3>, ..., <condition N>]}
\end{lstlisting}
\subsubsection{not}
The ``not" rule takes a \textbf{single} condition and returns the opposite of the value of that condition.
\begin{lstlisting}[language=json,firstnumber=1]
"if": {"not": <condition>}
\end{lstlisting}
\newpage
\subsection{Installation Rules}\label{section:installationrules}
Installation rules are individual entries in a ``build path" that can be executed. They consist of a name for the rule, and a list of named arguments that get passed to the rule. Rules do not have a common set of arguments; each rule defines its own set. The only common factor between rule specifications is the ``rule" keyword, which defines which rule is being selected.
\subsubsection{copy}
Copies an item from a location within the mod into the ``modfiles" folder of a mod. The ``copy" rule has two arguments:
\begin{itemize}
\item source
\item destination
\end{itemize}
``source" is the item within the ``modoptions" folder of a mod to be copied. It can be either a file or a directory. The path is separated by forward-slashes.\\
``destination" is the location within the ``modfiles" folder of a mod the source is to be copied to. It can be either a file or a directory. The path is separated by forward-slashes.\\\\
An example use of the rule to copy a file at ``modoptions/scripts/common\_scripts.txt" to ``modfiles/script64/t3004.txt" is given below.
\begin{lstlisting}[language=json,firstnumber=1]
{"rule": "copy": "source": "scripts/common_scripts.txt", "destination": "script64/t3004.txt"}
\end{lstlisting}
\newpage

\end{document}